% Team Note Sample Template
% These codes should be guaranteed, fast enough, short and easy to type.

\documentclass[landscape, 10pt, a4paper, oneside, twocolumn]{extarticle}
\usepackage{amssymb}
\usepackage{amsmath}
\usepackage{import}

\usepackage{teamnote}

\teamnote{PetrSU}{QA}{(Remeslennikov, Evstafeev, Titov)}

\ShowUsage
\ShowComplexity
\HideAuthor

\begin{document}

\maketitlepage

% Make Pagebreak if you want.
% \pagebreak 


\section{Graph}

\Algorithm
{Dinic}
{Almost linear in practice. $\mathcal{O}(m \sqrt n)$ on unit network.}
{$\mathcal{O}(n^{2}m)$}
{cpp}{source/Dinic.cpp}

\Algorithm
{Mincost}
{Complexity is strange but in practice works nice.}
{$\mathcal{O}(something\ big,\ never\ reached\ in\ ACM\ tasks)$}
{cpp}{source/Mincost.cpp}


% \Algorithm
% {General Matching}
% {Use \texttt{init} to init, \texttt{addEdge} to add edges, \texttt{match} to get matching, \texttt{Match} to find maximum matching. Vertices have 1-based index.}
% {$\mathcal{O}(VE)$}
% {cpp}{source/GeneralMatching.cpp}

\section{Data Structure}

\Algorithm
{Polynomial hashes}
{Almost unbreakable.}
{$\mathcal{O}(n), \mathcal{O}(1)$}
{cpp}{source/Hashes.cpp}


\section{Math}

\Algorithm
{Linear inverse modulo prime}
{Suprisingly laconic.}
{$\mathcal{O}(p)$}
{cpp}{source/LinearInverse.cpp}

\Algorithm
{FFT}
{You never know, you never know...}
{$\mathcal{O}(n \log n)$}
{cpp}{source/FFT.cpp}


% \section{Geometry}

% \Algorithm
% {Smallest Enclosing Circle}
% {Use \texttt{solve} with \texttt{vector<Point>}. It returns \texttt{Circle c}, \texttt{c.p} is center, \texttt{c.r} is radius.}
% {$\mathcal{O}(n)$}
% {cpp}{source/SmallestEnclosingCircle.cpp}


\end{document}




